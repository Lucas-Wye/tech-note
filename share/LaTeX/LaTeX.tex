
\documentclass{beamer}
\usepackage[UTF8,space,hyperref]{ctex}
\hypersetup{
    colorlinks=true,
    linkcolor=blue,
    filecolor=red,
    urlcolor=red,
}
% 使用等线字体
\setCJKsansfont{DengXian}

% for themes, etc.
\mode<presentation>
{ \usetheme{boxes} }

\usepackage{times}  % fonts are up to you
\usepackage{graphicx}

% these will be used later in the title page
\title{\LaTeX: What, Why and How}
\author{ QSC-Tech\ 舞王\\ lucas.zw.ye@outlook.com }

% 显示英文格式的日期
\CTEXoptions[today=old]
\date{March 7th, 2021}

% 去除下方导航栏
\setbeamertemplate{navigation symbols}{}

% have this if you'd like a recurring outline
\AtBeginSection[]  % "Beamer, do the following at the start of every section"
{
\begin{frame}<beamer> 
\frametitle{Outline} % make a frame titled "Outline"
\tableofcontents[currentsection]  % show TOC and highlight current section
\end{frame}
}

\begin{document}

% this prints title, author etc. info from above
\begin{frame}
\titlepage
\end{frame}

\section{What is \LaTeX}

\begin{frame}
\frametitle{Introduction}
\begin{itemize}
	\item a document preparation system for high-quality typesetting.
	\item for medium-to-large technical or scientific documents but also for almost any form of publishing.
	\item the standard for the communication and publication of scientific documents. 
\end{itemize}
\end{frame}

\begin{frame}
\frametitle{Introduction}
It is available as free software.

\LaTeX\ is not a word processor! Instead, \LaTeX\ encourages authors not to worry too much about the appearance of their documents but to concentrate on getting the right content.
\end{frame}

\begin{frame}
\frametitle{\LaTeX\ Features}
\begin{itemize}
	\item Typesetting journal articles, technical reports, books, and slide presentations.
	\item Control over large documents containing sectioning, cross-references, tables and figures.
	\item Typesetting of complex mathematical formulas.
	\item Automatic generation of bibliographies and indexes.
\end{itemize}
\end{frame}

\begin{frame}
\frametitle{Introduction}
\LaTeX\ is not a stand-alone typesetting program in itself, but document preparation software that runs on top of Donald E. Knuth's TeX typesetting system. 

TeX distributions usually bundle together all the parts needed for a working TeX system and they generally add to this both configuration and maintenance utilities.
\end{frame}

\begin{frame}
\frametitle{History}
\TeX\ is a computer language designed for use in typesetting.

Invented by Donald E. Knuth when he was revising the second volume of his multivolume magnum opus `The Art of Computer Programming`.

Bugs Reports: program bugs rise by powers of 2 each year from \$1.28 or so to a maximum of \$327.68.

\href{https://www.tug.org/whatis.html}{Click Here}
\end{frame}

\section{Why people use \LaTeX}
\begin{frame}
\frametitle{Why?}
Advantages
\begin{itemize}
	\item Focus on your content without worrying the format of your document.
	\item Helpful to process with a lot of files(texts, figures ...).
	\item High typographical quality of the documents.
	\item Documents with a lot of mathematics symbols.
	\item Easy to use `git` for version control.
\end{itemize}
Disadvantages
\begin{itemize}
	\item Take time to learn.
	\item Not easy to change the design of document.
\end{itemize}
\end{frame}

\section{How to learn and use \LaTeX}
\begin{frame}
\frametitle{How?}
Software requirements: 
\begin{itemize}
	\item `Texlive`, `MiKTex` or other useful LaTeX distributions. 
	\item Editor such as `Vim`.
	\item PDF Reader such as `SumatraPDF`.
	\item Or the Online Environment \href{overleaf.com/}{Overleaf}
\end{itemize}
My configuration: Texlive + Vim + \href{https://github.com/lervag/vimtex}{Vimtex} + SumatraPDF

Reference Link: \href{https://www.latex-project.org/get/}{\LaTeX\ Project}, \href{https://www.ctan.org/pkg/latex}{CTAN(Comprehensive \TeX\ Archive Network)}
\end{frame}

\begin{frame}
\frametitle{\LaTeX\ Tutorials}
\begin{itemize}
	\item \href{https://liam.page/2014/09/08/latex-introduction/}{ 一份其实很短的 LaTeX 入门文档 }
	\item \href{http://tug.ctan.org/info/lshort/english/lshort.pdf}{The Not So Short Introduction to \LaTeX 2$\epsilon$}\ or use the command `texdoc lshort`
\end{itemize}
\end{frame}

\begin{frame}
\frametitle{Use \LaTeX\ templates}
\begin{itemize}
	\item \href{http://latexstudio.net/}{Latex工作室}
	\item \href{https://www.overleaf.com/latex/templates}{Templates from Overleaf}
	\item \href{http://www.latextemplates.com/}{English LaTeX Templates}
	\item \href{https://github.com/TheNetAdmin/zjuthesis}{zjuthesis}
	\item \href{http://github.com/Lucas-Wye/Latex\_template}{LaTeX Templates for ZJU}
\end{itemize}
\end{frame}

\begin{frame}
	\centering Thanks
\end{frame}
\end{document}
